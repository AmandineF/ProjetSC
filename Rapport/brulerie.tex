\documentclass[11pt]{article}
\usepackage[T1]{fontenc}
\usepackage[utf8]{inputenc}
\usepackage{graphicx}

\usepackage[right=2.5cm, bottom=2.5cm,top=2.5cm, left=2.5cm]{geometry}
\title{\vspace{\fill} Cryptographie et sécurité \\ ~\textbf{IFT-606} \\~\\ Rapport de projet - La Brulerie}
\author{Amandine Fouillet - 14 130 638  ~\\ Frank Chassing - 14 153 710 ~\\ Thomas Signeux - 14 126 590}
\date{\today \vspace{\fill}}

\begin{document}
\maketitle
\newpage \thispagestyle{empty}
\null
\newpage
\tableofcontents
\newpage
\section*{Introduction}

Avec la multiplication des objets connectés, notamment les smartphones et autres appareils mobiles, de nombreux réseaux publics ont vu le jour dans diverses endroits. Cependant, la sécurité de ces réseaux est parfois douteuse et il est aisé de s’y introduire, à l'insu de tous. Une fois sur ces réseaux, les possibiltés d'intrusions et de malveillances sont nombreuses. 

Parmi ces intrusions, l'une d'elles peut être particulièrement efficace et dangereuse pour les personnes visées. Cette attaque s'appelle Man In The Middle.

Ce projet a pour but d'étudier cette faiblesse du réseau en développant une application capable de reproduire les principales attaques Man In the Middle mais également de pouvoir s'en protéger.

\section{Description du projet et des objectifs}

\subsection{L'attaque Man In The Middle}
L’attaque Man In The Middle a pour but d’intercepter les communications transitant entre deux machines, sans que ni l’une ni l’autre ne se doute que le canal de communication est compromis.
L’attaquant a alors la possibilité de lire mais aussi de modifier les messages  (dans une certaine mesure).

Il existe plusieurs techniques pour ce fair passer pour l'une ou l'autre de ces machines cibles.

\begin{itemize}
	\item l'ARP Spoofing : attaque la plus fréquente, utilisée dans la partie pratique de ce projet. On force les communications à transiter par l'ordinateur de l'attaquant qui se fait alors passer pour le routeur (gateway) du réseau.
	\item le DNS Poisoning : le but de cette attaque est de faire correspondre l'adresse IP d'une machine controlée par un pirate à un nom réel et valide d'une machine publique. Pour cela, il altère le ou les serveur(s) DNS du réseau.
	\item l'annalyse du trafic : technique de sniffing permettant de visualiser les informations non chiffrées.
	\item le déni de service : empêcher le fonctionnement d'une machine pour en prender le contrôle.
\end{itemize}


\subsection{Objectifs initiaux}
L’objectif de ce projet est de mettre en place des scénarios d’attaques de piratage de réseaux et de proposer des moyens de défense. Dans un premier temps, nous explorerons l'actualité des attaques de réseaux publics. Puis nous présenterons le travail réalisé lors du projet. Suivrons les chapitres liés aux descriptions techniques du projet ainsi qu'un guide d'utilisation.

\begin{itemize}
\item Pouvoir identifier les ordinateurs connectés au routeur
\item Couper l’ensemble des connexions au routeur afin de bénéficier d’une meilleure bande passante
\item Limiter la connexion des personnes connectées au réseau
\item Récupérer les informations transitant sur les réseau pour pouvoir les analyser et identifier les personnes se trouvant dans le bar
\item Organiser une défense aux attaques précédentes
\item Faire une vidéo de présentation 
\end{itemize}

\section{Présentation du travail réalisé}
Afin de répondre à nos différents objectifs nous avons décidé de mettre en place plusieurs attaques contre des utilisateurs connectés sur le même réseau que nous. Pour faciliter le déroulement des attaques nous avons créé une interface graphique simple qui permet à l'attaquant de mener des offensives contre sa victime en quelques clics. De façon à proposer un outil polyvalent, nous avons également introduit dans l'interface une fenêtre de défense avec deux différents types de protection implémentés. ~\\

Pour réaliser les attaques, la défense et l'interface graphique nous avons utilisé un seul langage de programmation, le langage python. Outre sa portabilité, ce langage offre une multitude de librairies déjà implémentées pour la manipulation de paquets réseau notamment le module Scapy que nous avons utilisé pour réaliser chacune de nos attaques et défenses. Ce module permet de forger, envoyer, réceptionner et manipuler des paquets réseau.~\\

Dans cette seconde section, nous allons vous présenter les scripts que nous avons implémentés, leur rôle et surtout leur fonctionnement.
\subsection{Attaque}
Nous avons implémenté trois différentes attaques : l'identification des utilisateurs sur le réseau, la coupure de ces usagers et enfin la récupération des liens internet qu'ils visitent. La première attaque est toujours effectuée car sans elle, il nous ait impossible d'effectuer les suivantes. Par contre les attaques de coupure et d'écoute, même si elles se ressemblent, ont été implémentées séparément. Ci dessous, nous allons décrire les différentes attaques, leur réalisation et leur fonctionnement à travers l'interface graphique.
\subsubsection{Identification}
Cette attaque consiste à analyser le réseau auquel l'attaquant est connecté et donner la liste de tous les utilisateurs présents et actifs sur ce même réseau. Ces usagers sont représentés par leur adresse IP et leur adresse MAC et ces identifiants seront utilisés pour cibler une victime lors des attaques suivantes. ~\\

\textbf{Identifier le réseau automatiquement}~\\

Toujours dans l'optique de permettre à l'utilisateur de communiquer seulement à travers une interface graphique, nous avons souhaité mettre en place une identification automatique du réseau ainsi que de la place d'adresse IP à scanner. Pour cela, nous avons eu recours à deux modules python : \textit{netifaces} et \textit{netaddr}. Grâce au premier, on peux parcourir la liste des interfaces de l'attaquant et récupérer celle qui est utilisée pour la connexion au réseau, puis on peut récupérer l'adresse IP locale de l'attaquant ainsi que le masque de sous-réseau. Puis, avec netaddr .. ~\\

\textbf{Envoie d'une requête}~\\

\textbf{Récupération des ordinateurs connectés}~\\

\subsubsection{Coupure}
\subsubsection{Sniffer}


\subsection{Défense}
Dans cette section, nous allons présenter nos applications de défense.
Afin de contrer les attaques ciblées sur une machine, nous avons déployé deux scripts. Le premier réalise une détection pour savoir si la machine est victime d'une attaque ARP spoofing. Le second script est une simple commande qui permet de contrer l'attaque de coupe de connexion.
\subsubsection{Détection}
\subsubsection{Anti-coupure}

\section{Guide utilisateur}
\subsection{Interface principale}
\subsection{Attaquer}
\subsubsection{Couper et rétablir}
\subsubsection{Sniffer}
\subsection{Se défendre}
\subsubsection{Détecter une attaque}
\subsubsection{Activer une protection contre les coupures}

\section{Planification et organisation}
\subsection{Outils utilisés}
\subsection{Planification}

\section{Améliorations}
A rediger 
identification : récupérer les noms des machines, l'os pour avoir plus d'informations sur les ordinateurs du réseau et pouvoir cibler une victime
coupure : ralentir au lieu de couper, possible et simple mais pas encore implémenté dans l'interface graphique
sniffer : obtenir plus d'informations sur ce que la victime est entrain de faire (site https, recherche google, mails, mot de passe), arrêter l'attaque man in the middle une fois l'écoute terminée (problème avec les threads)
détection : donner plus d'informations sur l'attaquant (nom de l'ordinateur, os..), faire tourner la détection en continu ? 
anti-cut : la désactivation pas très fonctionnelle (problème avec les thread)
\section*{Conclusion}

\newpage
\listoffigures
\end{document}
